% 1) pdflatex main
% 2) makeindex main.idx -s StyleInd.ist
% 3) biber main
% 4) pdflatex main x 2
\documentclass[11pt,fleqn]{book} % Default font size and left-justified equations
\usepackage[top=3cm,bottom=3cm,left=3.2cm,right=3.2cm,headsep=10pt,a4paper]{geometry} % Page margins
\usepackage{xcolor} % Required for specifying colors by name
\definecolor{ocre}{RGB}{243,102,25} % Define the orange color used for highlighting throughout the book
% Font Settings
\usepackage{avant} % Use the Avantgarde font for headings
%\usepackage{times} % Use the Times font for headings
\usepackage{mathptmx} % Use the Adobe Times Roman as the default text font together with math symbols from the Sym­bol, Chancery and Com­puter Modern fonts
\usepackage{microtype} % Slightly tweak font spacing for aesthetics
\usepackage[utf8]{inputenc} % Required for including letters with accents
\usepackage[T1]{fontenc} % Use 8-bit encoding that has 256 glyphs
\usepackage{amsmath,amssymb,amsfonts,latexsym}
\usepackage{latexsym}
\usepackage{amsmath}
\usepackage{amsfonts}
\usepackage[spanish]{babel}
\usepackage{mathrsfs}
\usepackage{psfrag}
\usepackage{graphicx}
\usepackage{amssymb}
\usepackage{multirow}
\usepackage{rotating}
\usepackage{enumerate}

\usepackage{cancel}

\usepackage{subfig}%PARA LAS FIGURAS MULTIPLES
\usepackage{multicol}


%\usepackage{enumitem}
%\usepackage[version=3]{mhchem}


\usepackage[spanish]{babel}

\spanishdecimal{.} %PARA EL PUNTO DECIMAL


% Index
\usepackage{calc} % For simpler calculation - used for spacing the index letter headings correctly
\usepackage{makeidx} % Required to make an index
\makeindex % Tells LaTeX to create the files required for indexing
%----------------------------------------------------------------------------------------

\input{structure} % Insert the commands.tex file which contains the majority of the structure behind the template


%DEFINICION DE LA CONSTANTE ELECTRICA
\newcommand{\ke}{ \frac{1}{4\pi \varepsilon_0 }}


%DEFINICION DEL VECTOR RADIAL UNITARIO PRIMADO
\newcommand{\rup}{ \mathbf{\hat{r}}\textsc{\textbf{'}} }






\begin{document}


%----------------------------------------------------------------------------------------
%	TITLE PAGE
%----------------------------------------------------------------------------------------
\begingroup
\thispagestyle{empty}
\AddToShipoutPicture*{\put(6,5){\includegraphics[scale=1]{background}}} % Image background
\centering
\vspace*{9cm}
\par\normalfont\fontsize{35}{35}\sffamily\selectfont
Notas de Electromagnetismo\\% Book title
\vspace*{1cm}
{\Huge Jos\'e Manuel D\'avila D\'avila}\par % Author name
\vspace*{1cm}
{\Huge David Torres Reyes}\par % Author name
\vspace*{1cm}
{\Huge Susana Valdez Alvarado}\par % Author name
\endgroup




%----------------------------------------------------------------------------------------
%	TABLE OF CONTENTS
%----------------------------------------------------------------------------------------
\chapterimage{Electro1} % Table of contents heading image
\pagestyle{empty} % No headers
\tableofcontents % Print the table of contents itself
\cleardoublepage % Forces the first chapter to start on an odd page so it's on the right
\pagestyle{fancy} % Print headers again
\tableofcontents

\chapter{Nombre del cap\'itulo}

\section{Nombre de la secci\'on}

Aqu\'i se escribe el texto
campo\index{Campo}

\begin{example}
 Este es el ambiente para escribir un ejemplo
\end{example}

\begin{definition}[Subespacio vectorial]
	Un subespacio vectorial ...
\end{definition}



	\begin{figure}[h!]
		\centering
		\includegraphics[height=0.5\linewidth]{pru}
		\caption{\emph{Descripci\'on de la imagen}}
		\label{fig:0Subvect}
	\end{figure}



\begin{theorem}[Desigualdad Cauchy Schwarz]
Si $  \mathbf{A} $ y  $\mathbf{B}$ son vectores de $\mathbb{V}$,
\end{theorem}
%

\begin{obs}
 El contenido de la observaci\'on
\end{obs}



%----------------------------------------------------------------------------------------
%	BIBLIOGRAPHY
%----------------------------------------------------------------------------------------
%----------------------------------------------------------------------------------------
%	BIBLIOGRAPHY
%----------------------------------------------------------------------------------------

\begin{thebibliography}{50}\label{ch:bib}
	%------------CAPITULO VI
	\bibitem{Wangness2001} Roald K. Wangness. Campos electromagn\'eticos. Editorial Limusa, S.A. de C.V (2001).

	\bibitem{Griffiths1999} Griffiths J. David. Introduction to electrodymanics. (pp. 446). 3rd ed. Prentice-Hall, Inc. (1999)

	\bibitem{Milford1986} John R. Reitz, \& Frederick J. Milford, \& Roert W. Christy. Fundamentos de la teor\'ia electromagn\'etica. (pp. 363-365). Tercera edici\'on. Addison-Wesley, Iberoamericana. S.A. (1986).


\end{thebibliography}



%----------------------------------------------------------------------------------------
%	INDEX
%----------------------------------------------------------------------------------------
\cleardoublepage
\setlength{\columnsep}{0.75cm}
\addcontentsline{toc}{chapter}{\textcolor{ocre}{\'Indice}}
\printindex



\end{document}
